
\section*{Legend}
$$
D = \text{Total damage over the fight}
$$
$$
T = \text{Fight length in seconds}
$$
$$
\text{DPS} = \frac{D}{T} = \text{damage per second}
$$
$$
S_x = \text{Total damage of all attacks of ability $x$}
$$
$$
A_j^x = \text{Attack damage of the $j^\text{th}$ attack of ability $x$,} \ j \in \{ 1, ..., N_x \}
$$
$$
N_x = \text{Number of attacks of ability $x$}
$$


\section*{Total DPS Equation}
%
The total damage over time will be the sum of all the random damage from each ability
%
$$
D = \sum_{x} S_x
$$
%
where for sword specialization, we have
%
$$
x = \{ \text{Melee Main-Hand, Melee Off-Hand, Sinister Strike, Eviscerate, Instant Poison VI} \}
$$
%
as well as Thunderfury procs for this with the weapon. Effects such as Slice n' Dice which do not directly cause damage are instead interpreted within this model as altering the the number of certain attacks $N_x$.

Melee attacks are considerably more complex than spells in terms of how their damage is calculated. We will begin by analyzing the base auto-attacks (`Melee Main-Hand', or $S_{mh}$ for short) to illustrate these details.

$$
S_{mh} = \sum_{j=1}^{N_{mh}} A_j^{mh}
$$
$$
= (1 - r) \sum_{j=1}^{N_{mh}} (B_j^{mh} + \frac{p}{14}) [1 - \mathbb{I}(\text{miss})] [1 - \mathbb{I}(\text{dodge})] [2 \mathbb{I}(\text{crit}) + 1] (G_j \mathbb{I}(\text{glancing blow}) + 1)
$$
%
where $\mathbb{I}$ is the indicator function for the random draw from the attack-type table (that is, there are not separate random draws for each event type). $r$ is the damage reduction coefficient, $B_j$ is the base random damage of the weapon attack before modification, and $G_j \sim \mathcal{U}(g_l, g_u)$ is the random amount of damage modification by the glancing blow effect. $p$ is the total attack power of the individual.

As usual for my style of calculators, the ultimate question is not ``what is my average DPS?'' but rather ``what is the largest value of DPS that I can perform with probability $c$'', with $c \rightarrow 0$ being a very uncommon result and $c \rightarrow 1$ being highly reliable. Notably, this requires calculation of higher moments than just the average, and stat weightings can effect these higher moments differently.





\section*{Expected Value of Melee}
%
$$
\E[S_{mh}] = \sum_{j=1}^{N_{mh}} \E \left[ A_j^{mh} \right]
$$
$$
= (1 - r) \E \left[ \sum_{j=1}^{N_{mh}} (B_j^{mh} + \frac{p}{14}) [1 - \mathbb{I}(\text{miss})] [1 - \mathbb{I}(\text{dodge})] [2 \mathbb{I}(\text{crit}) + 1] (G_j \mathbb{I}(\text{glancing blow}) + 1) \right]
$$
$$
= (1 - r) \sum_{j=1}^{N_{mh}} \E \left[ B_j^{mh} + \frac{p}{14} \right] \E \left\{ [1 - \mathbb{I}(\text{miss})] [1 - \mathbb{I}(\text{dodge})] [2 \mathbb{I}(\text{crit}) + 1] (G_j \mathbb{I}(\text{glancing blow}) + 1) \right\}
$$
$$
= (1 - r) \sum_{j=1}^{N_{mh}} (\E \left[ B_j^{mh} \right] + \frac{p}{14}) \left\{ 2 \E[ \mathbb{I}(\text{crit})] + \E[G_j] \E[\mathbb{I}(\text{glancing blow})] - \E[\mathbb{I}(\text{miss})] - \E[\mathbb{I}(\text{dodge})] + 1 \right\}
$$
%
(see Supplemental section for proof of above), which is then equal to
%
$$
= (1 - r) \sum_{j=1}^{N_{mh}} (\frac{d_l^{mh} + d_u^{mh}}{2} + \frac{p}{14}) \left[ 2 \P(\text{crit}) + \E[G_j] \P(\text{glancing blow}) - \P(\text{miss}) - \P(\text{dodge}) + 1 \right]
$$
%
where we used the facts that $B_j^{mh} \sim \mathcal{U}(d_l^{mh}, d_u^{mh})$ and $\E[\mathbb{I}(A)] = \P(A)$. Now parameterizing and expanding the probabilities as they relate to character stats,
%
$$
= (1 - \frac{a}{a + 5500}) N_{mh} (\frac{d_l^{mh} + d_u^{mh}}{2} + \frac{p}{14}) \left[ 2 \min\{q, 100\% - 40\% - \P(\text{miss}) - \P(\text{dodge}) \} + \frac{g_l + g_u}{2} 40\% \cdots \right.
$$
$$
\left. \cdots - \max\{ (5\% + [0.1\% \mathbb{I}(f \leq 10) + 0.2\% \mathbb{I}(f > 10)] f) 80\% + 20\% - (h - 1\%), 19\% \} \cdots \right.
$$
$$
\left. \cdots - 5\% + 0.1\% f + 1 \bigg] \right.
$$
%
where $a$ is the armor of the target, $f = (315 - \text{weapon skill})$, and $h$ is the total hit chance gained from talents and gear. Also,
%
$$
q = \max\{ q_* + \max\{q_\text{aura} - 1.8\%, 0\} - 3\%, 0 \}
$$
%
where $q_*$ is the base crit chance plus any gained from talents and agility, whereas $q_\text{aura}$ is crit gained directly from gear, buffs, and consumes. Also,
%
$$
g_l = \max\{1.3 - 0.05 f, 0.91\}
\qquad
g_u = \min\{ \max\{ 1.2 - 0.03 f, 0.99 \}, 0.2 \}
$$
%
There are several differences for on use abilities however, such as sinister strike and eviscerate. Namely,
%
$$
\P(\text{miss} \ | \ \text{ability}) = \max\{ 5\% + [0.1\% \mathbb{I}(f \leq 10) + 0.2\% \mathbb{I}(f > 10)] f - (h - 1\%), 0 \}
$$
$$
\P(\text{glancing blow} \ | \ \text{ability}) = 0
$$
%
and thus
%
$$
\P(\text{crit} \ | \ \text{ability}) = \min\{q, 100\% - \P(\text{miss} \ | \ \text{ability}) - \P(\text{dodge}) \}
$$
%
as well as the fact that the base weapon damage scaling inherits only from the main hand weapon.




\section*{Variability}
%
$$
\sigma^2 = \V[R_j] = \V\left[ (B_j + \frac{3}{3.5} s) H_j (2 C_j + 1) \right]
$$
$$
= (\V\left[ B_j + \frac{3}{3.5} s \right] + \E^2\left[ B_j + \frac{3}{3.5} s \right])
(\V\left[ H_j \right] + \E^2\left[ H_j \right])
(\V\left[ 2 C_j + 1 \right] + \E^2\left[ 2 C_j + 1 \right])
$$
$$
- \E^2\left[ B_j + \frac{3}{3.5} s \right] \E^2\left[ H_j \right] \E^2\left[ 2 C_j + 1 \right]
$$
$$
= (\V\left[ B_j \right] + ( \E\left[ B_j \right] + \frac{3}{3.5} s)^2)
(p (1 - p) + p^2)
( (1.5 + 0.5 \mathbb{I}(\text{ruin}))^2 \V\left[ C_j \right] + ((1.5 + 0.5 \mathbb{I}(\text{ruin})) \E\left[ C_j \right] + 1)^2)
$$
$$
- (\E\left[ B_j \right] + \frac{3}{3.5} s)^2 p^2 ((1.5 + 0.5 \mathbb{I}(\text{ruin})) \E\left[ C_j \right] + 1)^2
$$
$$
= (\frac{(b_k-a_k)^2}{12} + ( \frac{a_k + b_k}{2} + \frac{3}{3.5} s)^2)
p
( (1.5 + 0.5 \mathbb{I}(\text{ruin}))^2 q (1 - q) + ((1.5 + 0.5 \mathbb{I}(\text{ruin})) q + 1)^2)
$$
$$
- (\frac{a_k + b_k}{2} + \frac{3}{3.5} s)^2 p^2 ((1.5 + 0.5 \mathbb{I}(\text{ruin})) q + 1)^2
$$
%
where $a_j^k$ is the lower bound of the base shadow bolt damage of rank $k$ on the $j^\text{th}$ cast. Likewise, $b_j^k$ is this upper bound.



\section*{Convergence to a Central Limit}
%
If the fight is sufficiently long, the total DPS over the fight,
%
$$
\text{DPS} = \frac{D}{T}
\rightarrow \mathcal{N}(\frac{N}{T} \mu, \frac{N}{T^2} \sigma^2)
$$
%
where $N \sim \bigO(T)$ is the total scaling factor of the number of casts of spells such as $N_\text{sb}$, $N_\text{corr}$, and so on. The general question of DPS predictability becomes phrased as the value of $x$ such that
%
$$
\P(\text{DPS} > x) = c
$$
%
for a fixed value of 'reliability' $c \in [0,1]$. This can be calculated exactly for the Normal distribution as
%
$$
\P(\text{DPS} > x) = 1 - \P(\text{DPS} \leq x)
= 1 - F_\text{DPS}(x)
= 1 - \frac{1}{2} \left[ 1 + \text{erf}(\frac{ x - \frac{\mu}{T} }{ \frac{\sigma}{T} \sqrt{2} }) \right] = c
$$
%
$$
\frac{1}{2} \left[ 1 + \text{erf}(\frac{ x - \frac{\mu}{T} }{ \frac{\sigma}{T} \sqrt{2} }) \right] = 1 - c
$$
$$
x = \sqrt{2} \text{erf}^{-1} \left[ 2 (1 - c) - 1 \right] \frac{\sigma}{T} + \frac{\mu}{T}
$$
%
which provides the perpective that when we 'optimize' DPS in this sense, we are not only maximizing the average $\mu$ but also at some assigned cost to the variability through $\sigma$, which can be either penalizing $c > \frac{1}{2}$ or enhancing $c < \frac{1}{2}$. If the user does not wish to account for variance, the value at equality $c = \frac{1}{2}$ can be used to cancel out this contribution.



\section*{Solving for $N_\text{sb}$}
%
First note that if there is no life tapping needed for the fight, and no DoTs, we have
%
$$
N_\text{sb} = \left\lfloor \frac{T}{2.5} \right\rfloor
$$
%
but more generally, the value of $N_\text{sb}$ for a given fight length will depend on the maximum mana of the warlock, the total mana gained off the global cooldown from items such as mana potions, dark runes, and mp5 effects, and of course the minimal number of lifetaps required to maximize the number of shadow bolt casts on a given fight. Also relevant is the optional decision to include DoT spells, and if so, which. The strategy for including DoTs in the calculation is to pre-allocate the maximal number of full casts possible, and remove that required casting time and mana from the resources available for the shadow bolt filler.

This number of casts is determined exactly for a specific fight length $T$ via the following iterative method...
%
$$
\cdots
$$



\section*{DoT Clipping}

\subsection*{Curse of Agony (CoA)}
%
$$
\varepsilon_\text{CoA} = (? + s) \sum_{k=1}^{N_\text{CoA}} f(k)
\qquad
N_\text{CoA} = \left\lfloor \frac{\varepsilon_T}{24} \cdot 12 \right\rfloor = \left\lfloor \frac{\varepsilon_T}{2} \right\rfloor
$$
%
where $?$ is the base damage for that rank of CoA
%
$$
f(k) = 0.0541 \ \mathbb{I}(k \in \Z_1^4)
+ 0.0836 \ \mathbb{I}(k \in \Z_5^8)
+ 0.1123 \ \mathbb{I}(k \in \Z_9^{12})
$$

\subsection*{Corruption}
%
$$
\varepsilon_\text{cor} = \left\lfloor () \right\rfloor
$$



\subsection*{Supplemental: Immolate spell coefficients}
%
Hybrid spells are a bit trickier to calculate the spell coefficient for, so here is the work verifying the values used above... The over-time portion is calculated as
%
$$
OTP = \frac{1}{1 + \frac{1.5}{3.5}}
= (\frac{5}{3.5})^{-1}
= \frac{3.5}{5}
$$
%
and the direct portion is then
%
$$
DP = 1 - OTP = \frac{1.5}{5}
$$
%
and the final coefficients are then
%
$$
OTPC = OTP
$$
$$
DPC = \frac{1.5}{3.5} DP
= (\frac{1.5}{3.5})^2
$$




\section*{Supplemental: Raid-wide optimal choice of $c$?}
%
The choice of $c$ is often left to an individual, if that person wishes to have higher maximal parses, or more reliably higher average parses, etc... But there is a question of raid-wide DPS increase if each individual can be considered relatively homogeneous, is there an optimal choice for maximizing the total raid-wide damage?

The exact value for this is of course
%
$$
\text{D}_\text{raid} = \sum_{p=1}^{M_\text{DPS}} \text{D}_p
$$
%
where $\text{D}_p$ is the total damage of player $p$ out of the total $M_\text{DPS}$ DPS classes for a given fight. Similar to the previous argument of converge to a Central Limit in time over a long fight, we have potentially weaker convergence to a Central Limit over a large number of DPS players
%
$$
\text{D}_\text{raid} \rightarrow \mathcal{N}(M_\text{DPS} \frac{N}{T}, M_\text{DPS} \frac{N}{T^2})
$$
$$
\P(\text{DPS} > x) = 1 - \P(\text{DPS} \leq x)
= 1 - F_\text{DPS}(x)
= 1 - \frac{1}{2} \left[ 1 + \text{erf}(\frac{ x - \frac{\mu}{T} }{ \frac{\sigma}{T} \sqrt{2} }) \right] = c
$$
%
$$
\frac{1}{2} \left[ 1 + \text{erf}(\frac{ x - \frac{\mu}{T} }{ \frac{\sigma}{T} \sqrt{2} }) \right] = 1 - c
$$
$$
x = \sqrt{2} \text{erf}^{-1} \left[ 2 (1 - c) - 1 \right] \frac{\sigma}{T} + \frac{\mu}{T}
$$



\section*{Supplemental: calculation of expectation of attack table}
%
Let $X$ be
%
$$
X = [1 - \mathbb{I}(\text{miss})] [1 - \mathbb{I}(\text{dodge})] [2 \mathbb{I}(\text{crit}) + 1] (G_j \mathbb{I}(\text{glancing blow}) + 1) =
$$
$$
2 [G_j \mathbb{I}(\text{gb}) \mathbb{I}(\text{miss}) \mathbb{I}(\text{dodge}) \mathbb{I}(\text{crit}) - G_j \mathbb{I}(\text{gb}) \mathbb{I}(\text{miss}) \mathbb{I}(\text{crit}) - G_j \mathbb{I}(\text{gb}) \mathbb{I}(\text{dodge}) \mathbb{I}(\text{crit}) +  G_j \mathbb{I}(\text{gb}) \mathbb{I}(\text{crit}) \cdots
$$
$$
\cdots + \mathbb{I}(\text{miss}) \mathbb{I}(\text{dodge}) \mathbb{I}(\text{crit}) - \mathbb{I}(\text{miss}) \mathbb{I}(\text{crit}) - \mathbb{I}(\text{dodge}) \mathbb{I}(\text{crit}) + \mathbb{I}(\text{crit})] \cdots
$$
$$
\cdots + G_j \mathbb{I}(\text{gb}) \mathbb{I}(\text{miss}) \mathbb{I}(\text{dodge}) - G_j \mathbb{I}(\text{gb}) \mathbb{I}(\text{miss}) - G_j \mathbb{I}(\text{gb}) \mathbb{I}(\text{dodge}) + G_j \mathbb{I}(\text{gb}) + \mathbb{I}(\text{miss}) \mathbb{I}(\text{dodge}) \cdots
$$
$$
\cdots - \mathbb{I}(\text{miss}) - \mathbb{I}(\text{dodge}) + 1
$$
%
By using the fact that $\E[\mathbb{I}(A) \mathbb{I}(B)] = \E[\mathbb{I}(A \cap B)] = \P(A \cap B)$ and the fact that all of these events on the attack table are mutually exclusive, it is then shown that
%
$$
\E[X] = \E[ 2 \mathbb{I}(\text{crit}) + G_j \mathbb{I}(\text{gb}) - \mathbb{I}(\text{miss}) - \mathbb{I}(\text{dodge}) + 1]
$$
$$
= 2 \E[ \mathbb{I}(\text{crit})] + \E[G_j] \E[\mathbb{I}(\text{gb})] - \E[\mathbb{I}(\text{miss})] - \E[\mathbb{I}(\text{dodge})] + 1
$$


